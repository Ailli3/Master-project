% Options for packages loaded elsewhere
\PassOptionsToPackage{unicode}{hyperref}
\PassOptionsToPackage{hyphens}{url}
\PassOptionsToPackage{dvipsnames,svgnames,x11names}{xcolor}
%
\documentclass[
  12pt,
]{amsart}

\usepackage{amsmath,amssymb}
\usepackage{iftex}
\ifPDFTeX
  \usepackage[T1]{fontenc}
  \usepackage[utf8]{inputenc}
  \usepackage{textcomp} % provide euro and other symbols
\else % if luatex or xetex
  \usepackage{unicode-math}
  \defaultfontfeatures{Scale=MatchLowercase}
  \defaultfontfeatures[\rmfamily]{Ligatures=TeX,Scale=1}
\fi
\usepackage{lmodern}
\ifPDFTeX\else  
    % xetex/luatex font selection
\fi
% Use upquote if available, for straight quotes in verbatim environments
\IfFileExists{upquote.sty}{\usepackage{upquote}}{}
\IfFileExists{microtype.sty}{% use microtype if available
  \usepackage[]{microtype}
  \UseMicrotypeSet[protrusion]{basicmath} % disable protrusion for tt fonts
}{}
\makeatletter
\@ifundefined{KOMAClassName}{% if non-KOMA class
  \IfFileExists{parskip.sty}{%
    \usepackage{parskip}
  }{% else
    \setlength{\parindent}{0pt}
    \setlength{\parskip}{6pt plus 2pt minus 1pt}}
}{% if KOMA class
  \KOMAoptions{parskip=half}}
\makeatother
\usepackage{xcolor}
\setlength{\emergencystretch}{3em} % prevent overfull lines
\setcounter{secnumdepth}{-\maxdimen} % remove section numbering
% Make \paragraph and \subparagraph free-standing
\ifx\paragraph\undefined\else
  \let\oldparagraph\paragraph
  \renewcommand{\paragraph}[1]{\oldparagraph{#1}\mbox{}}
\fi
\ifx\subparagraph\undefined\else
  \let\oldsubparagraph\subparagraph
  \renewcommand{\subparagraph}[1]{\oldsubparagraph{#1}\mbox{}}
\fi


\providecommand{\tightlist}{%
  \setlength{\itemsep}{0pt}\setlength{\parskip}{0pt}}\usepackage{longtable,booktabs,array}
\usepackage{calc} % for calculating minipage widths
% Correct order of tables after \paragraph or \subparagraph
\usepackage{etoolbox}
\makeatletter
\patchcmd\longtable{\par}{\if@noskipsec\mbox{}\fi\par}{}{}
\makeatother
% Allow footnotes in longtable head/foot
\IfFileExists{footnotehyper.sty}{\usepackage{footnotehyper}}{\usepackage{footnote}}
\makesavenoteenv{longtable}
\usepackage{graphicx}
\makeatletter
\def\maxwidth{\ifdim\Gin@nat@width>\linewidth\linewidth\else\Gin@nat@width\fi}
\def\maxheight{\ifdim\Gin@nat@height>\textheight\textheight\else\Gin@nat@height\fi}
\makeatother
% Scale images if necessary, so that they will not overflow the page
% margins by default, and it is still possible to overwrite the defaults
% using explicit options in \includegraphics[width, height, ...]{}
\setkeys{Gin}{width=\maxwidth,height=\maxheight,keepaspectratio}
% Set default figure placement to htbp
\makeatletter
\def\fps@figure{htbp}
\makeatother

\usepackage{fullpage}
\usepackage{enumitem}
\makeatletter
\@ifpackageloaded{caption}{}{\usepackage{caption}}
\AtBeginDocument{%
\ifdefined\contentsname
  \renewcommand*\contentsname{Table of contents}
\else
  \newcommand\contentsname{Table of contents}
\fi
\ifdefined\listfigurename
  \renewcommand*\listfigurename{List of Figures}
\else
  \newcommand\listfigurename{List of Figures}
\fi
\ifdefined\listtablename
  \renewcommand*\listtablename{List of Tables}
\else
  \newcommand\listtablename{List of Tables}
\fi
\ifdefined\figurename
  \renewcommand*\figurename{Figure}
\else
  \newcommand\figurename{Figure}
\fi
\ifdefined\tablename
  \renewcommand*\tablename{Table}
\else
  \newcommand\tablename{Table}
\fi
}
\@ifpackageloaded{float}{}{\usepackage{float}}
\floatstyle{ruled}
\@ifundefined{c@chapter}{\newfloat{codelisting}{h}{lop}}{\newfloat{codelisting}{h}{lop}[chapter]}
\floatname{codelisting}{Listing}
\newcommand*\listoflistings{\listof{codelisting}{List of Listings}}
\makeatother
\makeatletter
\makeatother
\makeatletter
\@ifpackageloaded{caption}{}{\usepackage{caption}}
\@ifpackageloaded{subcaption}{}{\usepackage{subcaption}}
\makeatother
\ifLuaTeX
  \usepackage{selnolig}  % disable illegal ligatures
\fi
\usepackage{bookmark}

\IfFileExists{xurl.sty}{\usepackage{xurl}}{} % add URL line breaks if available
\urlstyle{same} % disable monospaced font for URLs
\hypersetup{
  pdftitle={Opimization of Row-Column Designs},
  pdfauthor={Jingning Yao},
  colorlinks=true,
  linkcolor={blue},
  filecolor={Maroon},
  citecolor={Blue},
  urlcolor={Blue},
  pdfcreator={LaTeX via pandoc}}

\title{Opimization of Row-Column Designs}
\author{Jingning Yao}
\date{2024-08-05}

\begin{document}

\section*{Masters Project Milestone Report}

\thispagestyle{empty}

\vspace{1em}

\noindent
\textbf{Name of the student:} Jingning Yao
\vspace{1em}

\noindent
\textbf{Name of the supervisor(s):} Dr.~Emi Tanaka \and Dr.~Katharine
Turner
\vspace{2em}

\noindent
\textbf{Title of the project (may be tentative):} Opimization of
Row-Column Designs
\vspace{3em}

\noindent
\textbf{Summary:} 
\subsection{ET comments}\label{et-comments}

\begin{itemize}
\tightlist
\item
  Break it in sections with clearer flow of the motivation and plan
\end{itemize}

\section{Introduction}\label{introduction}

\begin{itemize}
\tightlist
\item
  What is experimental design about?
\end{itemize}

Row-Column Designs refer to a kind of statistical experimental design
aiming to improve the efficiency of experiments and reducing the impact
from external factors like environment and unexpected adjacency.

We always want to ensure that when we are design experiments, they have
good capability of detecting the any differences between different
treatments. And we also want to know that whether these differences we
found come from treatments we give instead of other causes. While we
collect data from different treatments and replicates, we wish to find
the ideal one among our experiments, which depends on our goal.

\section{Background (or Literature
Review)}\label{background-or-literature-review}

\subsection{Completely randomised
design}\label{completely-randomised-design}

The most easy way to do it is the completely randomized design. It
doesn't have any restriction on allocation, so treatments are randomly
assigned to any position in the trial. In this case, unexpected
adjacency mentioned above may occur, which may be undesirable and cause
bias in the result.

\subsection{Randomised complete block
design}\label{randomised-complete-block-design}

So we need stratum to build the designs. A stratum is a restriction
imposed on the layout of design. It helps to adjust and accommodate for
field variation, and model its effect in a precise and theoretical way.
Observations are usually modeled by linear model. A basic design widely
discussed is randomized complete block design (RCB). The assumed model
is followed. \[Y_{ij} = \mu + p_i + \tau_j + \epsilon_{ij}\] Here
\(i=1,2,3,...,r\) and \(r\) is the number of replicate. And
\(j=1,2,3,...,\nu\) where \(\nu\) is the number of treatment.
\(\epsilon_{ij}\) is the the error of the model. It blocks every
replicates and each blocks contain every kinds of treatment.

It is possible to place all treatments in one replicate when the number
of treatment is small. When it come to larger number, larger number of
plot in a replicate is required, in which case resource limitation
should be considered and physical differences cannot be ignored.
Therefore it is necessary to consider incomplete block designs, which
only some treatments are placed in one block.

\subsection{Incomplete block design}\label{incomplete-block-design}

In incomplete block design, for all treatment having same status,
maximize the number of different pairwise comparisons in one block is
the optimal design we wish to obtain. So for RCB model, we can rewrite
in a matrices-vectors form as: \[Y=G\mu +R\rho+Z\beta+X\tau+\epsilon\]
Here \(R\) , \(Z\) and \(X\) are design matrices for replicates blocks
and treatments. \(G\) is a ones vector.

Through this model we are trying to find a optimal design for a
experiment.

\begin{itemize}
\tightlist
\item
  Citations needed.
\item
  For mathematical notations, use bold capital letters,
  e.g.~\(\textbf{G}\), for matrices and bold italic lower case for
  vectors, e.g.~\(\boldsymbol{b}\).
\end{itemize}

Here is a citation (\textbf{piephoNeighborBalanceEvenness2018?}). For a
bracket citation (\textbf{piephoNeighborBalanceEvenness2018?}).

\begin{itemize}
\tightlist
\item
  Describing A-optimality (or D-optimality).
\end{itemize}

\section{Plan}\label{plan}

\begin{itemize}
\tightlist
\item
  Your overall goal (broadly defined)
\item
  Explore the area of optimising row-column design based on A-optimality
  criterion with restriction of achieving the neighbourhood balance and
  eveness of distribution. Specifically, I will review
  (\textbf{piephoNeighborBalanceEvenness2018?}) and translate into a
  more mathematically rigorous framework.
\end{itemize}

\section{References}\label{references}



\vfill

\noindent
\textbf{Student's signature} \hfill \textbf{Supervisor's signature} \hspace{10em} \mbox{}\\

\noindent
\textbf{Date:} 2024-08-05\end{document}
