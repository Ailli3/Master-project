% Options for packages loaded elsewhere
\PassOptionsToPackage{unicode}{hyperref}
\PassOptionsToPackage{hyphens}{url}
%
\documentclass[
  a4paper,
  oneside,
  openany,
  12pt,
  onecolumn]{book}

\usepackage{amsmath,amssymb}
\usepackage{setspace}
\usepackage{iftex}
\ifPDFTeX
  \usepackage[T1]{fontenc}
  \usepackage[utf8]{inputenc}
  \usepackage{textcomp} % provide euro and other symbols
\else % if luatex or xetex
  \usepackage{unicode-math}
  \defaultfontfeatures{Scale=MatchLowercase}
  \defaultfontfeatures[\rmfamily]{Ligatures=TeX,Scale=1}
\fi
\usepackage{lmodern}
\ifPDFTeX\else  
    % xetex/luatex font selection
  \setmainfont[]{Public Sans}
\fi
% Use upquote if available, for straight quotes in verbatim environments
\IfFileExists{upquote.sty}{\usepackage{upquote}}{}
\IfFileExists{microtype.sty}{% use microtype if available
  \usepackage[]{microtype}
  \UseMicrotypeSet[protrusion]{basicmath} % disable protrusion for tt fonts
}{}
\makeatletter
\@ifundefined{KOMAClassName}{% if non-KOMA class
  \IfFileExists{parskip.sty}{%
    \usepackage{parskip}
  }{% else
    \setlength{\parindent}{0pt}
    \setlength{\parskip}{6pt plus 2pt minus 1pt}}
}{% if KOMA class
  \KOMAoptions{parskip=half}}
\makeatother
\usepackage{xcolor}
\usepackage[top=30mm,left=25mm,right=25mm,bottom=30mm]{geometry}
\setlength{\emergencystretch}{3em} % prevent overfull lines
\setcounter{secnumdepth}{2}
% Make \paragraph and \subparagraph free-standing
\ifx\paragraph\undefined\else
  \let\oldparagraph\paragraph
  \renewcommand{\paragraph}[1]{\oldparagraph{#1}\mbox{}}
\fi
\ifx\subparagraph\undefined\else
  \let\oldsubparagraph\subparagraph
  \renewcommand{\subparagraph}[1]{\oldsubparagraph{#1}\mbox{}}
\fi


\providecommand{\tightlist}{%
  \setlength{\itemsep}{0pt}\setlength{\parskip}{0pt}}\usepackage{longtable,booktabs,array}
\usepackage{calc} % for calculating minipage widths
% Correct order of tables after \paragraph or \subparagraph
\usepackage{etoolbox}
\makeatletter
\patchcmd\longtable{\par}{\if@noskipsec\mbox{}\fi\par}{}{}
\makeatother
% Allow footnotes in longtable head/foot
\IfFileExists{footnotehyper.sty}{\usepackage{footnotehyper}}{\usepackage{footnote}}
\makesavenoteenv{longtable}
\usepackage{graphicx}
\makeatletter
\def\maxwidth{\ifdim\Gin@nat@width>\linewidth\linewidth\else\Gin@nat@width\fi}
\def\maxheight{\ifdim\Gin@nat@height>\textheight\textheight\else\Gin@nat@height\fi}
\makeatother
% Scale images if necessary, so that they will not overflow the page
% margins by default, and it is still possible to overwrite the defaults
% using explicit options in \includegraphics[width, height, ...]{}
\setkeys{Gin}{width=\maxwidth,height=\maxheight,keepaspectratio}
% Set default figure placement to htbp
\makeatletter
\def\fps@figure{htbp}
\makeatother

\makeatletter
\@ifpackageloaded{tcolorbox}{}{\usepackage[skins,breakable]{tcolorbox}}
\@ifpackageloaded{fontawesome5}{}{\usepackage{fontawesome5}}
\definecolor{quarto-callout-color}{HTML}{909090}
\definecolor{quarto-callout-note-color}{HTML}{0758E5}
\definecolor{quarto-callout-important-color}{HTML}{CC1914}
\definecolor{quarto-callout-warning-color}{HTML}{EB9113}
\definecolor{quarto-callout-tip-color}{HTML}{00A047}
\definecolor{quarto-callout-caution-color}{HTML}{FC5300}
\definecolor{quarto-callout-color-frame}{HTML}{acacac}
\definecolor{quarto-callout-note-color-frame}{HTML}{4582ec}
\definecolor{quarto-callout-important-color-frame}{HTML}{d9534f}
\definecolor{quarto-callout-warning-color-frame}{HTML}{f0ad4e}
\definecolor{quarto-callout-tip-color-frame}{HTML}{02b875}
\definecolor{quarto-callout-caution-color-frame}{HTML}{fd7e14}
\makeatother
\makeatletter
\@ifpackageloaded{bookmark}{}{\usepackage{bookmark}}
\makeatother
\makeatletter
\@ifpackageloaded{caption}{}{\usepackage{caption}}
\AtBeginDocument{%
\ifdefined\contentsname
  \renewcommand*\contentsname{Table of contents}
\else
  \newcommand\contentsname{Table of contents}
\fi
\ifdefined\listfigurename
  \renewcommand*\listfigurename{List of Figures}
\else
  \newcommand\listfigurename{List of Figures}
\fi
\ifdefined\listtablename
  \renewcommand*\listtablename{List of Tables}
\else
  \newcommand\listtablename{List of Tables}
\fi
\ifdefined\figurename
  \renewcommand*\figurename{Figure}
\else
  \newcommand\figurename{Figure}
\fi
\ifdefined\tablename
  \renewcommand*\tablename{Table}
\else
  \newcommand\tablename{Table}
\fi
}
\@ifpackageloaded{float}{}{\usepackage{float}}
\floatstyle{ruled}
\@ifundefined{c@chapter}{\newfloat{codelisting}{h}{lop}}{\newfloat{codelisting}{h}{lop}[chapter]}
\floatname{codelisting}{Listing}
\newcommand*\listoflistings{\listof{codelisting}{List of Listings}}
\usepackage{amsthm}
\theoremstyle{plain}
\newtheorem{lemma}{Lemma}[chapter]
\theoremstyle{definition}
\newtheorem{definition}{Definition}[chapter]
\theoremstyle{remark}
\AtBeginDocument{\renewcommand*{\proofname}{Proof}}
\newtheorem*{remark}{Remark}
\newtheorem*{solution}{Solution}
\newtheorem{refremark}{Remark}[chapter]
\newtheorem{refsolution}{Solution}[chapter]
\makeatother
\makeatletter
\makeatother
\makeatletter
\@ifpackageloaded{caption}{}{\usepackage{caption}}
\@ifpackageloaded{subcaption}{}{\usepackage{subcaption}}
\makeatother
\ifLuaTeX
  \usepackage{selnolig}  % disable illegal ligatures
\fi
\usepackage[]{natbib}
\bibliographystyle{plainnat}
\usepackage{bookmark}

\IfFileExists{xurl.sty}{\usepackage{xurl}}{} % add URL line breaks if available
\urlstyle{same} % disable monospaced font for URLs
\hypersetup{
  pdftitle={ANU Thesis},
  pdfauthor={Your Name},
  hidelinks,
  pdfcreator={LaTeX via pandoc}}

\usepackage{fancyhdr}
\usepackage{graphicx}
\usepackage{multirow}
\usepackage{amsmath}	% Advanced maths commands
\usepackage{amssymb}	% Extra maths symbols
\usepackage{rotating} % Rotating table
\usepackage[toc,page]{appendix}
\usepackage{setspace}
\usepackage{rotating}
\usepackage{longtable}
\usepackage{natbib}
\usepackage{multirow}


\def\TODO#1{{\color{red}{\bf [TODO:} {\it{#1}}{\bf ]}}}

%%%%%% AUTHORS - PLACE YOUR OWN PACKAGES HERE %%%%%

\usepackage{color}
\usepackage[normalem]{ulem}

% These are autoref macros
\def\chapterautorefname{Chapter}
\def\sectionautorefname{Section}
\def\subsectionautorefname{Section}
\def\subsubsectionautorefname{Section}
\def\figureautorefname{Figure}
\def\tableautorefname{Table}
\def\equationautorefname{equation}

% There are editing macros
\newcommand{\red}[1]{{\textcolor{red}{#1}}}
\newcommand{\blue}[1]{{\textcolor{blue}{#1}}}
\newcommand{\cyan}[1]{{\textcolor{cyan}{#1}}}
\newcommand{\cut}[1]{{\red{\sout{#1}}}}
\newcommand{\add}[1]{{\cyan{#1}}}

% Autoref macros
\def\sectionautorefname{Section}
\def\subsectionautorefname{Section}
\def\subsubsectionautorefname{Section}
\def\figureautorefname{Figure}
\def\tableautorefname{Table}
\def\equationautorefname{equation}
\newcommand{\aref}[1]{\hyperref[#1]{Appendix~\ref{#1}}}

% Only include extra packages if you really need them. Common packages are:
%\usepackage{caption}
%\usepackage{subcaption}
%\captionsetup{compatibility=false} %https://tex.stackexchange.com/questions/31906/subcaption-package-compatibility-issue
\usepackage{multicol}        % Multi-column entries in tables
\usepackage{booktabs} %to include pandas latex tables
\usepackage{float}


\title{ANU Thesis}
\usepackage{etoolbox}
\makeatletter
\providecommand{\subtitle}[1]{% add subtitle to \maketitle
  \apptocmd{\@title}{\par {\large #1 \par}}{}{}
}
\makeatother
\subtitle{Quarto Template}
\author{Your Name}
\date{27th September 2024}

\begin{document}
  \begin{frontmatter}
  \begin{titlepage}
  %%% TITLE PAGE:
  \pagenumbering{roman}  % first use Roman numerals for page numbers

  \begin{titlepage}
    \begin{flushright}%
      \vspace{50mm}
      {\small A thesis submitted for the degree of {\it Doctor of
  Philosophy}}
      \rule[1ex]{\textwidth}{1pt}\\
      {\fontsize{9}{0} 27th September 2024}\\
      \vspace{25mm}
      {\fontsize{40}{44}\bfseries ANU Thesis\par}
        \vspace{12mm}
    	\parbox{\textwidth}{
  	\begin{flushright}
  		\fontsize{28}{30} Quarto Template
  	\end{flushright}}
  	    \vfill
      {\fontsize{20}{0}\bfseries Your Name}\\
      \vspace{2mm}
      {\fontsize{8}{0} Research School of Spectacular Sciences}\\
      \vspace{35mm}
      {\fontsize{10}{0}\bfseries supervised by}\\
      Prof.~Jane Smith, Dr.~John Smith
      
      \vspace{2.0cm}
  		\includegraphics[width=0.4\textwidth]{\_extensions/anu-thesis/assets/latex/ANU\_Primary\_Horizontal\_GoldBlack.eps}\\
   \end{flushright}%

   \clearpage\thispagestyle{empty}
   \normalfont
   \vspace*{\fill}
   \noindent
   \begin{tabular}{lp{10cm}}
     {\bf Doctor of Philosophy Thesis} & \\[2mm]
     {\bf Author:} & Your Name\\[2mm]
     {\bf Supervisors:} & Prof.~Jane Smith, Dr.~John Smith\\[2mm]
     
     {\bf Project period:} & 2024-01-01 -- 2028-01-01 \\[2mm]
   \end{tabular}\\[2mm]

  \noindent Research School of Spectacular Sciences\\
  \noindent Australian National University

  \end{titlepage}
  \setlength{\parindent}{0pt}
  \setlength{\parskip}{1ex plus 0.5ex minus 0.2ex}








  \end{titlepage}
  \end{frontmatter}


\renewcommand*\contentsname{Table of contents}
{
\setcounter{tocdepth}{2}
\tableofcontents
}
\phantomsection\addcontentsline{toc}{section}{List of Figures}
\listoffigures
\phantomsection\addcontentsline{toc}{section}{List of Tables}
\listoftables
\setstretch{1.2}
\mainmatter
\bookmarksetup{startatroot}

\chapter*{Preface}\label{preface}
\addcontentsline{toc}{chapter}{Preface}

\markboth{Preface}{Preface}

This thesis template is intended for honours, masters or PhD students at
the Australian National University (ANU) who wish to write their thesis
using the \href{https://quarto.org/}{Quarto} document format. It is
highly recommended for students who code using Python, R or Julia and
have many computational or analysis results in their thesis.

\begin{tcolorbox}[enhanced jigsaw, arc=.35mm, bottomtitle=1mm, bottomrule=.15mm, toprule=.15mm, colback=white, breakable, colbacktitle=quarto-callout-note-color!10!white, opacitybacktitle=0.6, coltitle=black, opacityback=0, colframe=quarto-callout-note-color-frame, titlerule=0mm, toptitle=1mm, title=\textcolor{quarto-callout-note-color}{\faInfo}\hspace{0.5em}{Note}, rightrule=.15mm, leftrule=.75mm, left=2mm]

This thesis template is available on GitHub at
\href{https://github.com/anuopensci/quarto-anu-thesis}{github.com/anuopensci/quarto-anu-thesis}.

\end{tcolorbox}

\section*{Benefits}\label{benefits}
\addcontentsline{toc}{section}{Benefits}

\markright{Benefits}

The benefits of using Quarto document include:

\begin{itemize}
\tightlist
\item
  It allows you to write your thesis in a simple markup language called
  \href{https://www.markdownguide.org/}{Markdown}. This means that you
  can focus on writing your thesis without having to worry about
  formatting.
\item
  The document can be output to a variety of formats including PDF,
  HTML, Word, and LaTeX.
\item
  Code can be easily embedded in the document and executed. This means
  that you can include the results of your analysis in your thesis
  without having to manually copy and paste them. This is a good
  reproducible and scientific practice.
\item
  You can easily integrate with aspects of GitHub (edit, reporting an
  issue, etc).
\end{itemize}

The above outlined benefits can also be considered as best practice.
Version controlling and collaborative writing (via Git and GitHub) are
important in managing multiple versions of your thesis and in
collaborating with your supervisory team. Embedding code in your thesis
is a good practice in reproducible research. Making your thesis in HTML
format can allow for interactive web elements to be embedded while PDF
format can be for general distribution and printing.

\section*{Getting started}\label{getting-started}
\addcontentsline{toc}{section}{Getting started}

\markright{Getting started}

There are several systems that you are expected to know to use this
template. These include:

\begin{itemize}
\tightlist
\item
  Markdown syntax for writing
\item
  Quarto or R Markdown syntax (note that these works for Python or Julia
  too) for embedding code
\item
  (Optional) Git and GitHub for hosting
\end{itemize}

\section*{Frequently asked questions}\label{frequently-asked-questions}
\addcontentsline{toc}{section}{Frequently asked questions}

\markright{Frequently asked questions}

\subsection*{What about Overleaf?}\label{what-about-overleaf}
\addcontentsline{toc}{subsection}{What about Overleaf?}

ANU has a professional account for Overleaf, which is great for those
that use LaTeX regularly. Unfortunately, there is no equivalent system
with track changes in Quarto. You can output the tex file from Quarto
document and use this in Overleaf. The changes made in this tex document
however has to be manually transferred back to the Quarto document. If
your main output is mainly mathematical and you have little to no code
outputs, Overleaf is probably a better choice.

\bookmarksetup{startatroot}

\chapter*{Abstract}\label{abstract}
\addcontentsline{toc}{chapter}{Abstract}

\markboth{Abstract}{Abstract}

\bookmarksetup{startatroot}

\chapter*{Acknowledgements}\label{acknowledgements}
\addcontentsline{toc}{chapter}{Acknowledgements}

\markboth{Acknowledgements}{Acknowledgements}

I would like to express my sincere gratitude to my dog, Chuckles, for
eating my research notes multiple times. If it wasn't for you, I would
have finished this thesis earlier.

\bookmarksetup{startatroot}

\chapter{Introduction}\label{sec-intro}

In the field of experimental design, efficient methods are crucial for
ensuring accurate and reliable results. One such method is the
row-column design, controlling variability in experiments involving two
factors, typically arranged in rows and columns. The row-column design
can be considered as an extension of the Latin square design with more
flexibility, allowing for different numbers of rows, columns, and
treatments. This flexibility makes row-column designs applicable to a
wider range of experimental settings.

To offer a row-column design that gives a precise estimation of
treatment effects, one way is to seek the optimal value of some
statistic criteria, for example, A-criteria(links to sections),
minimizing the variance of elementary treatment contrasts. Using linear
mixed model and assuming fixed treatment effects and random blocking
effect, \citet{butler2013optimal} has show the relation between
optimizing design and minimizing the value of A-criteria, and show some
possible algorithms to search optimal design in feasible set. These
algorithms mainly focus on comparing the arrangements of different
treatments, that is, doing permutations, and calculating their A values,
optimizing design by iterations.

However, some undesired cluster of replications or some treatment may
occur when algorithm are doing permutations along rows and columns.
\citet{piepho2018neighbor} found that such clustering is considered
undesirable by experimenters who worry that irregular environmental
gradients might negatively impact multiple replications of the same
treatment, potentially leading to biased treatment effect estimates.
Williams emphasis that there is a need to design a strategy to avoid
clustering and achieve even distribution of treatment replications among
the experimental field. Two properties of design are introduced. Even
distribution of treatment replications, abbreviated as ED, and neighbor
balance, abbreviated as NB. A good ED ensures every replications of a
treatment are widely spread in experimental field, and NB helps to avoid
replications of the some treatment cluster together repeatedly. Williams
introduce a scoring system to analysis ED and NB for a specific design,
and introduce a algorithm to optimize ED, NB and some average efficiency
factor can be represented by a specific statistic criteria.(maybe saying
some improvement is needed)

We offer an optimization strategy for a design problem, which we can
improving ED and NB during optimizing statistic criteria for a design,
and avoid unwanted clustering and self-adjacency on the resulting
design.In this algorithm, we use A-criteria to evaluate the efficiency
of a design. Before the algorithm, we randomly generate a design as an
initial design, and calculate the A-criteria as initial value. We update
design by selecting a better among its neighbors. The neighbors are
pair-wise permutations of a design. Typically, we select a neighbor from
all pairwise permutations of a design for iteration, but this does not
ensure ED and NB. To ensure ED and NB during optimization, we need to
add some constraints when generating the pairwise permutations.(maybe
explain what is the constraints) By filtering design with bad ED and NB,
we then optimize the statistic criteria of the design.

(sections in the thesis)

\bookmarksetup{startatroot}

\chapter{Background}\label{sec-bg}

\section{Linear model}\label{linear-model}

Suppose we have a linear model,

\begin{equation}\phantomsection\label{eq-lm}{\boldsymbol{y}=\mathbf{X}\boldsymbol{\tau} + \boldsymbol{\epsilon}}\end{equation}
where \(\boldsymbol{y}\) is \(n\times 1\) vector of \(n\) observations,
\(\boldsymbol{\tau}\) is a \(t\times 1\) vector of fixed effects,
\(\boldsymbol{\epsilon}\) is the \(n\times 1\) vector for error, and
\(\mathbf{X}\) is a design matrix has size \(n\times t\). We assume that
\(\boldsymbol{\epsilon} \sim N(\boldsymbol{0}, \sigma^2\boldsymbol{I}_{n})\)
and hence
\(\boldsymbol{y} \sim N(\mathbf{X}\boldsymbol{\tau}, \sigma^2\mathbf{I}_n)\).

The log-likelihood of Equation~\ref{eq-lm} is then given as:

\[
\log\ell(\boldsymbol{\tau};\boldsymbol{y}) = -\frac{n}{2}\log(2\pi)-n\log(\sigma)-\frac{1}{2\sigma^2}(\boldsymbol{y}-\boldsymbol{X}\boldsymbol{\tau})^\top(\boldsymbol{y}-\boldsymbol{X}\boldsymbol{\tau}).
\] The \((i,j)\)-th entry of the Fisher information matrix is defined as

\[
I_{ij}(\boldsymbol{\tau})=-\mathbb{E}\left(\frac{\partial^2}{\partial\tau_i\partial\tau_j}\log\ell(\boldsymbol{\tau};\boldsymbol{y})\right)
\] where \(\tau_i\) is the \(i\)-th entry of \(\boldsymbol{\tau}\).

\begin{lemma}[]\protect\hypertarget{lem-fim-lm}{}\label{lem-fim-lm}

The Fisher information matrix of Equation~\ref{eq-lm} is given as \[
\mathbf{C} = -\mathbb{E}\left(\frac{\partial^2}{\partial\boldsymbol{\tau}\partial\boldsymbol{\tau}^\top}\log\ell(\boldsymbol{\tau};\boldsymbol{y})\right)=\frac{1}{\sigma^2}\boldsymbol{X}^\top\boldsymbol{X}
\]

\end{lemma}

\begin{proof}
The second derivative of the log-likelihood function
\(\log\ell(\boldsymbol{\tau};\boldsymbol{y})\) is the Hessian matrix. We
have \[
\frac{\partial}{\partial\boldsymbol{\tau}}\log\ell(\boldsymbol{\tau};\boldsymbol{y})=\frac{1}{\sigma^2}\boldsymbol{X}^\top(\boldsymbol{y}-\boldsymbol{X}\boldsymbol{\tau})
\] and for second derivative is \[
\frac{\partial^2}{\partial\boldsymbol{\tau}\partial\boldsymbol{\tau}^\top}\log\ell(\boldsymbol{\tau};\boldsymbol{y})==-\frac{1}{\sigma^2}\boldsymbol{X}^\top\boldsymbol{X}
\] And in linear model assumption we have
\(\boldsymbol{y} \sim N(\mathbf{X}\boldsymbol{\tau}, \sigma^2\mathbf{I}_n)\)
and the Fisher information matrix is unbiased because, in the
expectation calculation process, we do not involve the randomness of
\(\boldsymbol{y}\). The Fisher information matrix is actually determined
by the design matrix \(\boldsymbol{X}\) and the error variance
\(\sigma^2\). Hence \[
\mathbb{E}\left(\frac{\partial^2}{\partial\boldsymbol{\tau}\partial\boldsymbol{\tau}^\top}\log\ell(\boldsymbol{\tau};\boldsymbol{y})\right)=-\frac{1}{\sigma^2}\boldsymbol{X}^\top\boldsymbol{X} = -\mathbf{C} 
\] So
\(\mathbf{C} = \frac{1}{\sigma^2}\boldsymbol{X}^\top\boldsymbol{X}\)
\end{proof}

\begin{lemma}[]\protect\hypertarget{lem-lm-var}{}\label{lem-lm-var}

The variance of the fixed effects for Equation~\ref{eq-lm} is equivalent
to the inverse of the Fisher information matrix, i.e.
\(var(\hat{\boldsymbol{\tau}})=\sigma^2(\boldsymbol{X}^\top\boldsymbol{X})^{-1} = \mathbf{C}^{-1}.\)

\end{lemma}

\begin{proof}
We know that the MLE of \(\boldsymbol{\tau}\) in a linear model is
\(\hat{\boldsymbol{\tau}}=(\boldsymbol{X}^\top\boldsymbol{X})^{-1}\boldsymbol{X}^\top\boldsymbol{y}\).
By assumption we have
\(\boldsymbol{y} \sim N(\mathbf{X}\boldsymbol{\tau}, \sigma^2\mathbf{I}_n)\).
So
\(\hat{\boldsymbol{\tau}}\sim N(\boldsymbol{\tau},\sigma^2(\boldsymbol{X}^\top\boldsymbol{X})^{-1})\).So
we have
\(var(\hat{\boldsymbol{\tau}}) = \sigma^2(\boldsymbol{X}^\top\boldsymbol{X})^{-1}\),
which is exactly the inverse of Fisher information matrix
\(\mathbf{C}^{-1}\).
\end{proof}

\section{Linear mixed model}\label{linear-mixed-model}

Linear mixed model extends linear model by incorporating additionally
incorporating random effects into the model that effectively give
greater flexibility and capability to incorporate known correlated
structures into the model. We now consider a linear mixed model
\begin{equation}\phantomsection\label{eq-lmm}{
\boldsymbol{y}=\boldsymbol{X}\boldsymbol{\tau}+\boldsymbol{Z}\boldsymbol{u}+\boldsymbol{\epsilon}
}\end{equation} here \(\boldsymbol{y}\) is \(n\times 1\) vector for
\(n\) observations, \(\boldsymbol{\tau}\) is a \(t\times1\) parameter
vector of treatment factors, \(\boldsymbol{u}\) is a \(q \times1\)
parameter vector of blocking effects, and \(\boldsymbol{\epsilon}\) is
the \(n\times 1\) error vector, \(\boldsymbol{X}\) and
\(\boldsymbol{Z}\) are design matrices of dimension \(n \times t\) and
\(n \times q\) for treatment factors and blocking factors, respectively.
We here assume blocking factors are random effect, with random error
\(\boldsymbol{\epsilon}\) we have \[
\begin{bmatrix}
\boldsymbol{u} \\
\boldsymbol{\epsilon} 
\end{bmatrix}
\sim
N\left(
\begin{bmatrix}
\boldsymbol{0} \\
\boldsymbol{0}
\end{bmatrix}
,
\begin{bmatrix}
\boldsymbol{G} & \mathbf{0} \\
\mathbf{0} & \boldsymbol{R}
\end{bmatrix}
\right),
\] where \(\boldsymbol{G}\) is the \(q \times q\) variance matrix for
\(\boldsymbol{u}\) and \(\boldsymbol{R}\) is \(n\times n\) variance
matrix for \(\boldsymbol{\epsilon}\).

\subsection{A-criterion}\label{a-criterion}

Optimizing the A-value is crucial in row-column designs, for it directly
relates to the precision of the treatment effect estimates. The A-value,
a measure of design efficiency, quantifies how well the experimental
design minimizes the variability when estimating treatment effects.

By focusing on minimizing the A-value, we aim to achieve a design that
provides the most precise estimates of treatment effects. A lower
A-value means that the design is more efficient, leading to smaller
variances for the difference between treatment effect estimations.

We first start with a simple example, that is, we consider treatment
factors \(\boldsymbol{\tau}\) are fixed, to elucidate the influence of
A-criterion.

basing on assumption, we have
\begin{equation}\phantomsection\label{eq-asp}{
\boldsymbol{y}=\boldsymbol{X}\boldsymbol{\tau}+\boldsymbol{Z}\boldsymbol{u}+\boldsymbol{\epsilon}\sim N(\boldsymbol{X}\boldsymbol{\tau},\boldsymbol{R}+\boldsymbol{ZGZ}^\top)
}\end{equation} So for objective function, we can write out the
distributions

\[
\boldsymbol{y}|\boldsymbol{u};\boldsymbol{\tau},\boldsymbol{R}\sim N(\boldsymbol{X}\boldsymbol{\tau}+\boldsymbol{Z}\boldsymbol{u},\boldsymbol{R}) \\
\boldsymbol{u}\sim N(\boldsymbol{0},\boldsymbol{G})
\]

We want to give a precise estimation on \(\boldsymbol{\tau}\). As we
mentioned, we have the distribution for response variable
\(\boldsymbol{y}\sim\) We can use generalized least squares(GLS) by
rewrite the model as: \[
\boldsymbol{y} = \boldsymbol{X}\boldsymbol{\tau} + \zeta
\] Here
\(\zeta = \boldsymbol{Z}\boldsymbol{u}+\boldsymbol{\epsilon}\sim N(0, \boldsymbol{R}+\boldsymbol{Z}\boldsymbol{G}\boldsymbol{Z}^\top)\).
\citet{henderson1975best} shows that the GLS estimation of
\(\boldsymbol{\tau}\) is any solution to \[
\boldsymbol{X}^\top\boldsymbol{V}^{-1}\boldsymbol{X}\hat{\boldsymbol{\tau}}_{gls}=\boldsymbol{X}^\top\boldsymbol{V}^{-1}\boldsymbol{y}
\] Here
\(\boldsymbol{V}=\boldsymbol{R}+\boldsymbol{Z}\boldsymbol{G}\boldsymbol{Z}^\top\).
So
\(\hat{\boldsymbol{\tau}}_{gls} = (\boldsymbol{X}^\top\boldsymbol{V}^{-1}\boldsymbol{X})^{-1}\boldsymbol{X}^\top\boldsymbol{V}^{-1}\boldsymbol{y}\)

\citet{henderson1959estimation} emphasis that computing matrix
\(\boldsymbol{V}\) which is often large is difficult. So here we use
joint log likelihood.

From \citet{butler2013optimal}, we conduct a maximum log likelihood by
following objective function: \[
\log f_Y(\boldsymbol{y}|\boldsymbol{u};\boldsymbol{\tau},\boldsymbol{R})+\log f_u(\boldsymbol{u};\boldsymbol{G})
\] :::\{\#lem-joint-density-lmm\} So log of joint density is given as

\begin{align*}
\mathscr{L}&=\log f_Y(\boldsymbol{y}|\boldsymbol{u};\boldsymbol{\tau},\boldsymbol{R})+\log f_u(\boldsymbol{u};\boldsymbol{G})\\
&=-\frac{1}{2}\left(\log|\boldsymbol{R}|+\log|\boldsymbol{G}|+(\boldsymbol{y}-\boldsymbol{X}\boldsymbol{\tau}-\boldsymbol{Z}\boldsymbol{u})^\top \mathbf{R}^{-1}(\boldsymbol{y}-\boldsymbol{X}\boldsymbol{\tau}-\boldsymbol{Z}\boldsymbol{u})+\boldsymbol{u}^\top\boldsymbol{G}^{-1}\boldsymbol{u}\right)
\end{align*}

:::

\begin{proof}
We have density function for
\(\boldsymbol{y}|\boldsymbol{u};\boldsymbol{\tau},\boldsymbol{R}\sim N(\boldsymbol{X}\boldsymbol{\tau}+\boldsymbol{Z}\boldsymbol{u},\boldsymbol{R})\)

\[
f_y = \frac{1}{\sqrt{(2\pi)^{n}|\boldsymbol{R}|}}exp(-\frac{1}{2}(\boldsymbol{y}-(\boldsymbol{X}\boldsymbol{\tau}+\boldsymbol{Z}\boldsymbol{u}))^\top\boldsymbol{R}^{-1}(\boldsymbol{y}-(\boldsymbol{X}\boldsymbol{\tau}+\boldsymbol{Z}\boldsymbol{u})))
\] And density function for \(\boldsymbol{u}\) \[
f_u = \frac{1}{\sqrt{(2\pi)^{l}|\boldsymbol{G}|}}exp(-\frac{1}{2}\boldsymbol{u}^\top\boldsymbol{G}^{-1}\boldsymbol{u})
\] Ignoring constant part, we have \[
\log f_y=-\frac{1}{2}[\ln |\boldsymbol{R}|+(\boldsymbol{y}-(\boldsymbol{X}\boldsymbol{\tau}+\boldsymbol{Z}\boldsymbol{u}))^\top\boldsymbol{R}^{-1}(\boldsymbol{y}-(\boldsymbol{X}\boldsymbol{\tau}+\boldsymbol{Z}\boldsymbol{u}))]
\] \[
\log f_u = -\frac{1}{2}[\ln |\boldsymbol{G}+\boldsymbol{u}^\top\boldsymbol{G}^{-1}\boldsymbol{u}]
\] So we have our log of joint density function.
\end{proof}

We determine that
\(\frac{\partial\mathscr{L}}{\partial\boldsymbol{\tau}}=\frac{\partial\mathscr{L}}{\partial\boldsymbol{u}}=\boldsymbol{0}\),
and write the equation into a matrix form
\begin{equation}\phantomsection\label{eq-ll}{
\begin{bmatrix}
\boldsymbol{X}^\top\boldsymbol{R}^{-1}\boldsymbol{X} & \boldsymbol{X}^\top\boldsymbol{R}^{-1}\boldsymbol{Z}\\
\boldsymbol{Z}^\top\boldsymbol{R}^{-1}\boldsymbol{X} & \boldsymbol{Z}^\top\boldsymbol{R}^{-1}\boldsymbol{Z}+ \boldsymbol{G}^{-1}
\end{bmatrix}
\begin{bmatrix}
\hat{\boldsymbol{\tau}}_{llm}\\
\hat{\boldsymbol{u}}_{llm}
\end{bmatrix}=
\begin{bmatrix}
\boldsymbol{X}^\top\boldsymbol{R}^{-1}\boldsymbol{y}\\
\boldsymbol{Z}^\top\boldsymbol{R}^{-1}\boldsymbol{y}
\end{bmatrix}
}\end{equation}

Let \[
\boldsymbol{C}=
\begin{bmatrix}
\boldsymbol{X}^\top\boldsymbol{R}^{-1}\boldsymbol{X} & \boldsymbol{X}^\top\boldsymbol{R}^{-1}\boldsymbol{Z}\\
\boldsymbol{Z}^\top\boldsymbol{R}^{-1}\boldsymbol{X} & \boldsymbol{Z}^\top\boldsymbol{R}^{-1}\boldsymbol{Z}+ \boldsymbol{G}^{-1}
\end{bmatrix} 
\quad
\hat{\boldsymbol{\beta}}_{llm}=\begin{bmatrix}
\hat{\boldsymbol{\tau}}_{llm}\\
\hat{\boldsymbol{u}}_{llm}
\end{bmatrix}
\quad
\boldsymbol{W}=\begin{bmatrix}\boldsymbol{X} &\boldsymbol{Z}\end{bmatrix}
\] By cancelling \(\boldsymbol{u}\), we have \[
\boldsymbol{X}^\top\boldsymbol{R}^{-1}\boldsymbol{X}\boldsymbol{\tau}+\boldsymbol{X}^\top\boldsymbol{R}^{-1}\boldsymbol{Z}(\boldsymbol{Z}^\top\boldsymbol{R}^{-1}\boldsymbol{Z}+\boldsymbol{G}^{-1})\boldsymbol{Z}^\top\boldsymbol{R}^{-1}y=\boldsymbol{X}^\top\boldsymbol{R}^{-1}y\\
\] \[
\Rightarrow \boldsymbol{X}^\top[\boldsymbol{R}^{-1}-\boldsymbol{R}^{-1}\boldsymbol{Z}(\boldsymbol{Z}^\top\boldsymbol{R}^{-1}\boldsymbol{Z}+\boldsymbol{G}^{-1})^{-1}\boldsymbol{Z}^\top\boldsymbol{R}^{-1}]\boldsymbol{X}\boldsymbol{\tau}=\boldsymbol{X}^\top[\boldsymbol{R}^{-1}-\boldsymbol{R}^{-1}\boldsymbol{Z}(\boldsymbol{Z}^\top\boldsymbol{R}^{-1}\boldsymbol{Z}+\boldsymbol{G}^{-1})^{-1}\boldsymbol{Z}^\top\boldsymbol{R}^{-1}]y\\
\] \[
\Rightarrow \boldsymbol{X}^\top\boldsymbol{P}\boldsymbol{X}\boldsymbol{\tau}=\boldsymbol{X}^\top\boldsymbol{P}\boldsymbol{y}
\] where
\(\boldsymbol{P}=\boldsymbol{R}^{-1}-\boldsymbol{R}^{-1}\boldsymbol{Z}(\boldsymbol{Z}^\top\boldsymbol{R}^{-1}\boldsymbol{Z}+\boldsymbol{G}^{-1})^{-1}\boldsymbol{Z}^\top\boldsymbol{R}^{-1}\),
let
\(\boldsymbol{C}_{11}=\boldsymbol{X}^\top\boldsymbol{P}\boldsymbol{X}\)
then we have the form similar to GLS estimation for
\(\hat{\boldsymbol{\tau}}\), which is \[
\boldsymbol{C}_{11}\hat{\boldsymbol{\tau}}_{llm}=\boldsymbol{X}^\top\boldsymbol{P}\boldsymbol{y}
\] and the estimation of \(\boldsymbol{\tau}\) is
\(\hat{\boldsymbol{\tau}}_{llm}=\boldsymbol{C}_{11}^{-1}\boldsymbol{X}^\top\boldsymbol{P}\boldsymbol{y}\),
which is equivalent with GLS estimation

\begin{proof}
We only need to prove that \(\boldsymbol{P}=\boldsymbol{V}^{-1}\).
\begin{align*}
\boldsymbol{P}\boldsymbol{V} &=(\boldsymbol{R}^{-1}-\boldsymbol{R}^{-1}\boldsymbol{Z}(\boldsymbol{Z}^\top\boldsymbol{R}^{-1}\boldsymbol{Z}+\boldsymbol{G}^{-1})^{-1}\boldsymbol{Z}^\top\boldsymbol{R}^{-1})(\boldsymbol{R}+\boldsymbol{Z}\boldsymbol{G}\boldsymbol{Z}^\top)\\
&= \boldsymbol{I} + \boldsymbol{R}^{-1}\boldsymbol{Z}\boldsymbol{G}\boldsymbol{Z}^\top - \boldsymbol{R}^{-1}\boldsymbol{Z}(\boldsymbol{Z}^\top\boldsymbol{R}^{-1}\boldsymbol{Z}+\boldsymbol{G}^{-1})^{-1}\boldsymbol{Z}^\top \\
&\quad \quad \quad \quad \quad \quad\quad\quad -\boldsymbol{R}^{-1}\boldsymbol{Z}(\boldsymbol{Z}^\top\boldsymbol{R}^{-1}\boldsymbol{Z}+\boldsymbol{G}^{-1})^{-1}\boldsymbol{Z}^\top\boldsymbol{R}^{-1}\boldsymbol{Z}\boldsymbol{G}\boldsymbol{Z}^\top\\
&=\boldsymbol{I} + \boldsymbol{R}^{-1}\boldsymbol{Z}\boldsymbol{G}\boldsymbol{Z}^\top -\boldsymbol{R}^{-1}\boldsymbol{Z}(\boldsymbol{Z}^\top\boldsymbol{R}^{-1}\boldsymbol{Z}+\boldsymbol{G}^{-1})^{-1}\boldsymbol{Z}^\top(\boldsymbol{I}+\boldsymbol{R}^{-1}\boldsymbol{Z}\boldsymbol{G}\boldsymbol{Z}^\top)\\
&=\boldsymbol{I} + \boldsymbol{R}^{-1}\boldsymbol{Z}\boldsymbol{G}\boldsymbol{Z}^\top -\boldsymbol{R}^{-1}\boldsymbol{Z}(\boldsymbol{Z}^\top\boldsymbol{R}^{-1}\boldsymbol{Z}+\boldsymbol{G}^{-1})^{-1}(\boldsymbol{Z}^\top+\boldsymbol{Z}^\top\boldsymbol{R}^{-1}\boldsymbol{Z}\boldsymbol{G}\boldsymbol{Z}^\top)\\
&= \boldsymbol{I} + \boldsymbol{R}^{-1}\boldsymbol{Z}\boldsymbol{G}\boldsymbol{Z}^\top -\boldsymbol{R}^{-1}\boldsymbol{Z}(\boldsymbol{Z}^\top\boldsymbol{R}^{-1}\boldsymbol{Z}+\boldsymbol{G}^{-1})^{-1}(\boldsymbol{I}+\boldsymbol{Z}^\top\boldsymbol{R}^{-1}\boldsymbol{Z}\boldsymbol{G})\boldsymbol{Z}^\top\\
&= \boldsymbol{I} + \boldsymbol{R}^{-1}\boldsymbol{Z}\boldsymbol{G}\boldsymbol{Z}^\top -\boldsymbol{R}^{-1}\boldsymbol{Z}(\boldsymbol{Z}^\top\boldsymbol{R}^{-1}\boldsymbol{Z}+\boldsymbol{G}^{-1})^{-1}(\boldsymbol{G}^{-1}+\boldsymbol{Z}^\top\boldsymbol{R}^{-1}\boldsymbol{Z})\boldsymbol{G}\boldsymbol{Z}^\top\\
&= \boldsymbol{I}+\boldsymbol{R}^{-1}\boldsymbol{Z}\boldsymbol{G}\boldsymbol{Z}^\top-\boldsymbol{R}^{-1}\boldsymbol{Z}\boldsymbol{G}\boldsymbol{Z}^\top\\
& = \boldsymbol{I}
\end{align*} So \(\hat{\boldsymbol{\tau}}_{llm}\) and
\(\hat{\boldsymbol{\tau}}_{gls}\) are equivalent, and we denote them as
\(\hat{\boldsymbol{\tau}}\)
\end{proof}

Now we have our estimation for the treatment factor, and experimental
design aims to further refine our design by focusing on the precision of
these estimates. Specifically, we aim to optimize the design so that the
treatment effects are estimated with minimal variance, ensuring that the
differences between any two treatment levels are as small as possible.
To achieve this, we introduce the A-value as a criterion for evaluating
the design.

\begin{definition}[]\protect\hypertarget{def-A-value}{}\label{def-A-value}

Basing on the model formula Equation~\ref{eq-lmm}, and a estimation of
treatment factor \(\hat{\boldsymbol{\tau}}\) has \(n_{\tau}\) factors.
A-criterion measure the average predicted error variance of different
treatments. Let
\(V_{ij}= var(\hat{\tau}_i-\hat{\tau}_j)=var(\hat{\tau}_i)+var(\hat{\tau}_j)-2cov(\hat{\tau}_i,\hat{\tau}_j)\),
and a A-value \(\mathscr{A}\) is
\begin{equation}\phantomsection\label{eq-a}{
\mathscr{A}=\frac{1}{n_{\tau}(n_{\tau}-1)}\sum_{i}\sum_{j<i}V_{ij}
}\end{equation}

\end{definition}

To discover the relationship between the A-value and the design matrix,
I need to find the variance-covariance matrix of
\(\hat{\boldsymbol{\tau}}\). In fact, it can be proof that
\(\boldsymbol{\tau}-\hat{\boldsymbol{\tau}}\sim N(0,\boldsymbol{C}_{11}^{-1})\).

(lemma and proof)

From Equation~\ref{eq-ll} and Equation~\ref{eq-asp}, we have
\(\hat{\boldsymbol{\tau}}\sim N(\boldsymbol{\tau},\boldsymbol{X}^\top\boldsymbol{R}^{-1}(\boldsymbol{R}+\boldsymbol{ZGZ}^\top)(\boldsymbol{X}^\top\boldsymbol{R}^{-1})^\top)\)

We denote \begin{align*}
\boldsymbol{M} &= \boldsymbol{X}^\top \boldsymbol{R}^{-1} \boldsymbol{X}\\
\boldsymbol{N} &= \boldsymbol{X}^\top \boldsymbol{R}^{-1} \boldsymbol{Z}\\
\boldsymbol{J} &= \boldsymbol{Z}^\top \boldsymbol{R}^{-1} \boldsymbol{X}\\
\boldsymbol{K} &= \boldsymbol{Z}^\top \boldsymbol{R}^{-1} \boldsymbol{Z} + \boldsymbol{G}^{-1}\\
\end{align*} In the context of row-column design, the \(\boldsymbol{K}\)
matrix is invertible. Schur complement of \(\boldsymbol{K}\) is \[
\boldsymbol{S} = \boldsymbol{M} - \boldsymbol{N} \boldsymbol{K}^{-1} \boldsymbol{J}
\] And the inverse matrix of \(\boldsymbol{C}\) can be written as
\begin{equation}\phantomsection\label{eq-iv}{
\boldsymbol{C}^{-1}=
\begin{bmatrix}
\boldsymbol{C}^{11} & \boldsymbol{C}^{12} \\
\boldsymbol{C}^{21} & \boldsymbol{C}^{22}
\end{bmatrix}=
\begin{bmatrix}
\boldsymbol{S}^{-1} & -\boldsymbol{S}^{-1} \boldsymbol{N} \boldsymbol{K}^{-1} \\
-\boldsymbol{K}^{-1} \boldsymbol{J} \boldsymbol{S}^{-1} & \boldsymbol{K}^{-1} + \boldsymbol{K}^{-1} \boldsymbol{J} \boldsymbol{S}^{-1} \boldsymbol{N} \boldsymbol{K}^{-1}
\end{bmatrix}
}\end{equation} So from Equation~\ref{eq-ll} we have \[
\begin{bmatrix}
\hat{\boldsymbol{\tau}} \\
\hat{\boldsymbol{u}}_{llm}
\end{bmatrix}
=
\begin{bmatrix}
\boldsymbol{C}^{11} & \boldsymbol{C}^{12} \\
\boldsymbol{C}^{21} & \boldsymbol{C}^{22}
\end{bmatrix}
\begin{bmatrix}
\boldsymbol{X}^\top\boldsymbol{R}^{-1}\boldsymbol{y}\\
\boldsymbol{Z}^\top\boldsymbol{R}^{-1}\boldsymbol{y}
\end{bmatrix}
=
\begin{bmatrix}
\boldsymbol{U}_1\\
\boldsymbol{U}_2
\end{bmatrix}\boldsymbol{y}
\] Here \begin{align*}
&\boldsymbol{U}_1 = \boldsymbol{C}^{11}\boldsymbol{X}^\top\boldsymbol{R}^{-1}+\boldsymbol{C}^{12}\boldsymbol{Z}^\top\boldsymbol{R}^{-1}\\
&\boldsymbol{U}_2 = \boldsymbol{C}^{21}\boldsymbol{X}^\top\boldsymbol{R}^{-1}+\boldsymbol{C}^{22}\boldsymbol{Z}^\top\boldsymbol{R}^{-1}
\end{align*}

From Equation~\ref{eq-ll} and Equation~\ref{eq-asp}, we have
\(\hat{\boldsymbol{\tau}}\sim N(\boldsymbol{\tau},\boldsymbol{U}_1(\boldsymbol{R}+\boldsymbol{ZGZ}^\top)\boldsymbol{U}_1^\top)\)

And we have following results

\begin{align*}
\begin{bmatrix}
\boldsymbol{U}_1 \\
\boldsymbol{U}_2
\end{bmatrix}
\begin{bmatrix}
\boldsymbol{X} & \boldsymbol{Z}
\end{bmatrix}
&=
\begin{bmatrix}
\boldsymbol{C}^{11} & \boldsymbol{C}^{12} \\
\boldsymbol{C}^{21} & \boldsymbol{C}^{22}
\end{bmatrix}
\begin{bmatrix}
\boldsymbol{X}^\top\boldsymbol{R}^{-1}\\
\boldsymbol{Z}^\top\boldsymbol{R}^{-1}
\end{bmatrix}
\begin{bmatrix}
\boldsymbol{X} & \boldsymbol{Z}
\end{bmatrix}\\
&=
\begin{bmatrix}
\boldsymbol{C}^{11} & \boldsymbol{C}^{12} \\
\boldsymbol{C}^{21} & \boldsymbol{C}^{22}
\end{bmatrix}
\begin{bmatrix}
\boldsymbol{X}^\top\boldsymbol{R}^{-1}\boldsymbol{X} & \boldsymbol{X}^\top\boldsymbol{R}^{-1}\boldsymbol{Z}\\
\boldsymbol{Z}^\top\boldsymbol{R}^{-1}\boldsymbol{X} & \boldsymbol{Z}^\top\boldsymbol{R}^{-1}\boldsymbol{Z}
\end{bmatrix}\\
&=
\begin{bmatrix}
\boldsymbol{C}^{11} & \boldsymbol{C}^{12} \\
\boldsymbol{C}^{21} & \boldsymbol{C}^{22}
\end{bmatrix}
(
\begin{bmatrix}
\boldsymbol{X}^\top\boldsymbol{R}^{-1}\boldsymbol{X} & \boldsymbol{X}^\top\boldsymbol{R}^{-1}\boldsymbol{Z}\\
\boldsymbol{Z}^\top\boldsymbol{R}^{-1}\boldsymbol{X} & \boldsymbol{Z}^\top\boldsymbol{R}^{-1}\boldsymbol{Z}+\boldsymbol{G}^{-1}
\end{bmatrix}-
\begin{bmatrix}
0 & 0 \\
0 & \boldsymbol{G}^{-1}
\end{bmatrix})\\
&=
\boldsymbol{I}-
\begin{bmatrix}
0 & -\boldsymbol{C}^{12}\boldsymbol{G}^{-1} \\
0 & -\boldsymbol{C}^{22}\boldsymbol{G}^{-1}
\end{bmatrix}\\
&=
\begin{bmatrix}
\boldsymbol{I} & -\boldsymbol{C}^{12}\boldsymbol{G}^{-1} \\
0 & \boldsymbol{I}-\boldsymbol{C}^{22}\boldsymbol{G}^{-1}
\end{bmatrix}
\end{align*}

So we have \begin{align*}
\boldsymbol{U}_1\boldsymbol{X}&=\boldsymbol{I}\\
\boldsymbol{U}_1\boldsymbol{Z}&=-\boldsymbol{C}^{12}\boldsymbol{G}^{-1}\\
\end{align*} For the variance of estimation we have \begin{align*}
var(\hat{\boldsymbol{\tau}})
&=
\boldsymbol{U}_1(\boldsymbol{R}+\boldsymbol{ZGZ}^\top)\boldsymbol{U}_1^\top\\
&= \boldsymbol{U}_1\boldsymbol{R}\boldsymbol{U}_1^\top+\boldsymbol{U}_1\boldsymbol{ZGZ}^\top\boldsymbol{U}_1^\top\\
&= (\boldsymbol{C}^{11}\boldsymbol{X}^\top\boldsymbol{R}^{-1}+\boldsymbol{C}^{12}\boldsymbol{Z}^\top\boldsymbol{R}^{-1})\boldsymbol{R}\boldsymbol{U}_1^\top+\boldsymbol{U}_1\boldsymbol{ZGZ}^\top\boldsymbol{U}_1^\top\\
&=\boldsymbol{C}^{11}\boldsymbol{X}^\top\boldsymbol{U}_1^\top+\boldsymbol{C}^{12}\boldsymbol{Z}^\top\boldsymbol{U}_1^\top+\boldsymbol{C}^{12}\boldsymbol{G}^{-1}\boldsymbol{G}(\boldsymbol{G}^{-1})^\top(\boldsymbol{C}^{12})^\top\\
&=\boldsymbol{C}^{11}-\boldsymbol{C}^{12}(\boldsymbol{G}^{-1})^\top(\boldsymbol{C}^{12})^\top+\boldsymbol{C}^{12}(\boldsymbol{G}^{-1})^\top(\boldsymbol{C}^{12})^\top\\
&=\boldsymbol{C}^{11}
\end{align*}

What is \(\boldsymbol{C}^{11}\)? what is the relation between
\(\boldsymbol{C}_{11}\) and \(\boldsymbol{C}^{11}\)? From
Equation~\ref{eq-iv} we have \(\boldsymbol{C}^{11}=\boldsymbol{S}^{-1}\)
and \(\boldsymbol{S}\) is \[
\boldsymbol{S} = \boldsymbol{X}^\top \boldsymbol{R}^{-1} \boldsymbol{X} - \boldsymbol{X}^\top \boldsymbol{R}^{-1} \boldsymbol{Z} (\boldsymbol{Z}^\top \boldsymbol{R}^{-1} \boldsymbol{Z} + \boldsymbol{G}^{-1})^{-1} \boldsymbol{Z}^\top \boldsymbol{R}^{-1} \boldsymbol{X}
\] And base on the complement of \(\boldsymbol{C}_{11}\), we rewrite the
\(\boldsymbol{S}\) \[
\begin{align*}
\boldsymbol{S}
&=\boldsymbol{X}^\top(\boldsymbol{R}^{-1} - \boldsymbol{X}^\top \boldsymbol{R}^{-1} \boldsymbol{Z} (\boldsymbol{Z}^\top \boldsymbol{R}^{-1} \boldsymbol{Z} + \boldsymbol{G}^{-1})^{-1} \boldsymbol{Z}^\top \boldsymbol{R}^{-1})\boldsymbol{X}\\
&=\boldsymbol{X}^\top\boldsymbol{P}\boldsymbol{X}\\
&=\boldsymbol{C}_{11}
\end{align*}
\] So
\(var(\hat{\boldsymbol{\tau}})=\boldsymbol{C}^{11}=\boldsymbol{C}_{11}^{-1}\).

So we have
\(\boldsymbol{\tau}-\hat{\boldsymbol{\tau}}\sim N(0,\boldsymbol{C}_{11}^{-1})\).To
examine a specific form of \(\boldsymbol{\tau}\), in general case, we do
linear transform on \(\boldsymbol{\tau}\):
\(\hat{\boldsymbol{\pi}}=\boldsymbol{D}\hat{\boldsymbol{\tau}}\), where
\(\boldsymbol{D}\) is some transform matrix, so we have
\(\boldsymbol{D}(\boldsymbol{\tau}-\hat{\boldsymbol{\tau}})=\boldsymbol{\pi}-\hat{\boldsymbol{\pi}}\sim N(0,\boldsymbol{D}\boldsymbol{C}_{11}^{-}\boldsymbol{D}^\top)\).
We denote
\(\boldsymbol{\Lambda}=\boldsymbol{D}\boldsymbol{C}_{11}^{-}\boldsymbol{D}^\top\).
If \(\boldsymbol{D}\) is identical matrix \(\boldsymbol{I}\), then
\(\boldsymbol{\Lambda}=\boldsymbol{C}_{11}^{-1}\) and
\(\hat{\boldsymbol{\pi}}=\hat{\boldsymbol{\tau}}\).

We know that A-criterion is the mean of predicted error variance of the
parameter from Equation~\ref{eq-a} i.e.~ \[
\mathscr{A}=\frac{1}{n_{\tau}(n_{\tau}-1)}\sum_{i}\sum_{j<i}V_{ij}
\] Having variance-covariance matrix
\(\boldsymbol{\Lambda}=\boldsymbol{C}_{11}^{-1}\), we can rewrite the
sum part as
\(n_{\tau}tr(\boldsymbol{\Lambda})-\mathbb{1}_{n_{\tau}}^\top\boldsymbol{\Lambda}\mathbb{1}_{n_{\tau}}\).So
we have \begin{equation}\phantomsection\label{eq-an}{
\mathscr{A}=\frac{1}{n_{\tau}(n_{\tau}-1)}[n_{\tau}tr(\boldsymbol{\Lambda})-\mathbb{1}_{n_{\tau}}^\top\boldsymbol{\Lambda}\mathbb{1}_{n_{\tau}}]
}\end{equation} same result from \citet{butler2013model}

Derivation above indicate that
\(\mathscr{A}\propto tr(\boldsymbol{\Lambda})\), A-criterion as the mean
of predicted error variance of the parameter, we prefer it as small as
possible to obtain a accurate result from experiment, which means the
trace of virance-covirance matrix \(\boldsymbol{\Lambda}\) should be as
small as possible. And this is our goal on optimal experimental design.

\section{Neighbor balance and eveness of
distribution}\label{neighbor-balance-and-eveness-of-distribution}

\subsection{Concepts of NB and ED}\label{concepts-of-nb-and-ed}

\citet{piepho2018neighbor} emphasis the the concepts of neighbor balance
and even distribution are crucial to mitigating biases and ensuring the
reliability of results in row-column design.

Neighbor balance (NB) refers to the principle that, in a row-column
experimental design, the frequency with which two treatments are
adjacent or near each other should not be excessively high. High
adjacency frequency between two treatments can lead to mutual influence,
which may cause bias to the experimental results. For example, if the
effect of one treatment can spread to neighboring areas, frequent
adjacency could interfere with accurate measurement of each treatment's
true effect next to it. Therefore, it is essential to control the
adjacency frequency of different treatments to prevent high adjacency
for two specific treatments.

Even distribution(ED) aims to ensure that different replications of the
same treatment are widely distributed across the experimental field,
rather than being clustered in a specific area. This strategy helps to
avoid biases caused by specific environmental conditions in certain
parts of the experiment field. If replications of one treatment are over
concentrated in one area, unique environmental factors in that area
might affect the treatment's performance, leading to biased
observations. By evenly distributing replications, environmental
interference can be minimized, so that we can enhance the reliability of
the experimental results.

(maybe some example plots or pictures)

\subsection{Measuring NB and ED}\label{measuring-nb-and-ed}

\subsubsection{Evaluating NB with adjacency
matrix}\label{evaluating-nb-with-adjacency-matrix}

In \citet{piepho2018neighbor}, there is a assumption that they are
optimizing a binary design, which means each treatment appears only once
in each row and column. Under this assumption, their balancing mechanism
considers diagonal adjacency, Knight moves, and even more distant points
to ensure an optimal balance. In my optimization process, I begin with a
randomly selected design matrix. Consequently, my approach considers not
only diagonal adjacency but also the adjacent points directly above,
below, to the left, and to the right.

I use an adjacency matrix to count the number of times each treatment is
adjacent to another. This matrix serves as a crucial tool in my
optimization process, enabling precise tracking and adjustment of
treatment placements to achieve neighbor balance.

We denote the adjacency matrix as \(\boldsymbol{A}\), and for treatment
\(i\) and \(j\) in treatment set \(T\) \(\boldsymbol{A}_{ij}\)
represents the count of times treatment \(i\) is adjacent to treatment
\(j\). Here ``adjacent'' means treatment \(j\) is located next to
treatment \(i\) (maybe a picture to show it)

For Given design \(\mathcal{D}\) and \(\mathcal{D}_{r,c}\) represents
the treatment at row \(r\) and column \(c\). So \(\boldsymbol{A}_{ij}\)
can be expressed as:

\[
\boldsymbol{A}_{ij}=\sum_{r=1}^{R}\sum_{c=1}^{C}I_{r,c}(i) F_{r,c}(j)
\] where \[
F_{r,c}(j)=
\sum_{m \in \{-1,0,1\}}\sum_{n \in \{-1,1\}}I_{r+m,c+n}(j)+\sum_{m \in \{-1,1\}}I_{r+m,c}(j)
\] \(R\) and \(C\) are total number of rows and columns and
\(I_{r,c}(\cdot)\) is the indicator function, which takes value under
following cases \[
I_{r,c}(i)=
\begin{cases}
1 & \text{if } \mathcal{D}_{r,c}=i \\
0 & \text{if } \mathcal{D}_{r,c}\neq i & \text{or } r<1,r>R,c<1,c>C\\
\end{cases}
\]

The function \(F_{r,c}(j)\) here is actually counting the times that
treatment \(j\) occurs at places around the position row \(r\) and
column \(c\).

We measure NB by taking the maximum of the elements in adjacency matrix
\(\boldsymbol{A}\). Our NB criteria is \[
C_{NB}=max\{\boldsymbol{A}_{ij}\}-min\{\boldsymbol{A}_{ij}\}  \quad i,j\in T
\]

\subsubsection{Evaluating ED with minimum row and column
span}\label{evaluating-ed-with-minimum-row-and-column-span}

The goal of evaluating the evenness of distribution (ED) is to find the
row and column spans for treatments across the entire design matrix. We
would like this value as large as possible This ensures that the
treatments \(t\in T\) is distributed as evenly as possible within the
rows and columns, reducing clustering and promoting a balanced design.

The row span for a given treatment \(t \in T\) is defined as the
difference between the maximum and minimum row indices where t appears
in experiment field. \[
RS(t)=max\{r:\mathcal{D}_{r,c}=t\}-min\{r:\mathcal{D}_{r,c}=t\} \quad 1<r<R,\quad 1<c<C
\] And the minimum row span of a design \(\mathcal{D}\) is \[
MRS(\mathcal{D})=min\{RS(t)\},\quad t \in T
\] Same for column span \[
CS(t)=max\{c:\mathcal{D}_{r,c}=t\}-min\{c:\mathcal{D}_{r,c}=t\} \quad 1<r<R,\quad 1<c<C
\] \[
MCS(\mathcal{D})=min\{CS(t)\},\quad t \in T
\] So, for the changes in the design matrix \(\mathcal{D}\) during the
search process, we tend to accept only those changes where the Minimum
Row Span (\(MRS\)) and Minimum Column Span (\(MCS\)) remain the same or
become smaller.

\bookmarksetup{startatroot}

\chapter{Methods}\label{sec-methods}

\section{Modeling row-column design}\label{modeling-row-column-design}

As mentioned in previous chapter, we use a linear mixed model (LMM) to
model the row-column design having two distinct sources of variation,
typically referred to as ``row'' and ``column'' factors. This design
structure appears frequently in agricultural and industrial trials,
where treatments are applied across units organized in a grid-like
pattern, and both row and column effects may influence the outcomes.

In my assumptions, the row and column effects are treated as random
effects, which means that they are random factors for spatial or
systematic factors across different rows and columns of the experiment
field. The treatment effects, on the other hand, are treated as fixed
effects because they represent the primary factors of interest that we
wish to evaluate in terms of their influence on the response variable.

Recalling Equation~\ref{eq-lmm}, the treatment effects are modeled as
fixed effects, represented by the treatment design matrix
\(\boldsymbol{X}\) with parameter vector \(\boldsymbol{\tau}\),
measuring the influence of each treatment on the response variable. The
matrix \(\boldsymbol{X}\) is constructed such that each row corresponds
to an experimental unit, and indicators in each column indicates whether
a treatment is applied or not. Detailed structure will be shown in
section(?????)

The random effects are modeled through the matrix \(\boldsymbol{Z}\) and
parameter vector \(\boldsymbol{u}\). Matrix \(\boldsymbol{Z}\) is
designed to capture the row and column structure of the experimental
field, in which entries represent the position of each experimental unit
located in some specific rows and columns. The parameters in vector
\(\boldsymbol{u}\) are corresponding row and column effects. They are
assumed to follow a normal distribution with mean zero and
variance-covariance matrix \(\boldsymbol{G}\).

With \(\boldsymbol{y}\) as the vector of observed responses and
\(\boldsymbol{\epsilon}\) as error term, a row-column design can be
modeled by Equation~\ref{eq-lmm} . where \(\boldsymbol{X\tau}\)
represents the fixed treatment effects and \(\boldsymbol{Zu}\) captures
the random variations basing on rows and columns.

\subsection{Random effects matrix}\label{random-effects-matrix}

The design matrix for the random effects, which is row and column in my
linear mixed model follows a binary indicator structure. For example, we
a have a \(4\times 4\) experiment field, having \(4\) rows, \(4\)
columns and \(16\) units. Then random effects matrix should be a
\(16 \times 8\) matrix containing binary indicator as the one presented.
\[
\begin{bmatrix}
\begin{array}{cccc|cccc}
1 & 0 & 0 & 0 & 1 & 0 & 0 & 0 \\
0 & 1 & 0 & 0 & 1 & 0 & 0 & 0 \\
0 & 0 & 1 & 0 & 1 & 0 & 0 & 0 \\
0 & 0 & 0 & 1 & 1 & 0 & 0 & 0 \\
1 & 0 & 0 & 0 & 0 & 1 & 0 & 0 \\
0 & 1 & 0 & 0 & 0 & 1 & 0 & 0 \\
0 & 0 & 1 & 0 & 0 & 1 & 0 & 0 \\
0 & 0 & 0 & 1 & 0 & 1 & 0 & 0 \\
1 & 0 & 0 & 0 & 0 & 0 & 1 & 0 \\
0 & 1 & 0 & 0 & 0 & 0 & 1 & 0 \\
0 & 0 & 1 & 0 & 0 & 0 & 1 & 0 \\
0 & 0 & 0 & 1 & 0 & 0 & 1 & 0 \\
1 & 0 & 0 & 0 & 0 & 0 & 0 & 1 \\
0 & 1 & 0 & 0 & 0 & 0 & 0 & 1 \\
0 & 0 & 1 & 0 & 0 & 0 & 0 & 1 \\
0 & 0 & 0 & 1 & 0 & 0 & 0 & 1 \\
\end{array}
\end{bmatrix}
\] Each row corresponds to a specific experimental unit, while the
columns represent the row and column factors in the experimental
layout.In this matrix, the first set of columns represents the column
effects, while the second set of columns represents the row effects.
Each entry in this matrix is binary, where a value of 1 indicates that
the experimental unit belongs to a specific row or column, and a 0
indicates otherwise. For example, the first row of the matrix has a 1 in
both first and fifth columns, meaning that the corresponding unit of it
is in the first column and the first row. This structure ensures that
each unit is uniquely associated with one row and one column, and we can
model the random effects accordingly.

In a more general case, suppose we have a \(m\times n\) row-column
experiment field. We should have a random effect matrix with \(mn\) rows
and \(m+n\) columns with binary numbers. In this paper, we assume that
the row effects and column effects are independent with each other.
However, in more complex experimental design cases, they may be
potentially correlated. The design of the random effects matrix, which
separates the row and column effects as independent variables,
simplifies the modeling process and the analysis of potential
correlations between these effects in more advanced settings. This
structure allows for easier identification and analysis of interactions
between row and column effects, making the model flexible and adaptable
to different levels of complexity in experimental designs.

\subsection{Design matrix for
treatments}\label{design-matrix-for-treatments}

The design matrix for the treatment effects is constructed to capture
the influence of each treatment on the response variable. In a
row-column experimental design, each experimental unit is assigned a
specific treatment.The entries in design matrix represents these
assignments using binary indicators. Like random effects matrix each row
in the matrix corresponds to an experimental unit, while each column
represents a different treatment. Here we still use \(4\times 4\)
experiment field as example, and suppose we have \(4\) different
treatments for each have \(4\) replications, needing \(16\) experiment
unite. An example design matrix \(\boldsymbol{X}\) for treatments should
be a \(16\times 4\) matrix with binary indicators as shown below

\[
\begin{bmatrix}
1 & 0 & 0 & 0\\
0 & 1 & 0 & 0\\
0 & 0 & 1 & 0\\
0 & 0 & 0 & 1\\
0 & 1 & 0 & 0\\
1 & 0 & 0 & 0\\
0 & 0 & 1 & 0\\
0 & 0 & 0 & 1\\
0 & 0 & 1 & 0\\
0 & 1 & 0 & 0\\
1 & 0 & 0 & 0\\
0 & 0 & 0 & 1\\
0 & 0 & 0 & 1\\
0 & 0 & 1 & 0\\
0 & 1 & 0 & 0\\
1 & 0 & 0 & 0\\
\end{bmatrix}
\] For a given experimental unit, that is a given row, the design matrix
contains a 1 in the column corresponding to the treatment applied to
that unit, and 0 elsewhere.This structure allows for a clear and
efficient representation of which treatment is applied to each unit. For
example, the first row of the example design matrix represents the first
treatment is applied in the unit locating on the first column, first
row.

if there are \(t\) treatments and \(N\) experimental units, the design
matrix will have \(N\) rows and \(t\) columns. Then the design matrix
for treatment \(\boldsymbol{X}\) with size \(N\times t\), should satisfy
that for any row \(n_{i}\) \[
\sum_{j=1}^{t} \boldsymbol{X}_{n_{i},j}=1
\] That is, there is only one treatment can be applied in each
experiment unit. And for any treatment \(t_j\) with \(r_j\)
replications, it has \[
\sum_{i=1}^{N}\boldsymbol{X}_{i,t_j}=r_j
\] All replications of a treatment are applied in experimental field.

\subsection{Assumptions for A-value
calculation}\label{assumptions-for-a-value-calculation}

It is important to clarify the key assumptions made in this study before
calculating A value for a row-column design. Recalling
Equation~\ref{eq-an}, for calculating A value we need the covariance
matrix for the random effects, matrix \(\boldsymbol{G}\), covariance
matrix for the error term, matrix \(\boldsymbol{R}\), transformation
matrix \(\boldsymbol{D}\), random effect matrix \(\boldsymbol{Z}\) and
treatment design matrix \(\boldsymbol{X}\). We need some basic setup for
these matrices.

Assume that we now have a row-column matrix with \(R\) rows, \(C\)
columns and \(RC\) plots.

For the covariance matrix for the random effects, matrix
\(\boldsymbol{G}\), which captures the variability introduced by the row
and column effects. I assume it is a \((R+C)\times(R+C)\) diagonal
matrix, that is, it has following form,
\#\#\#\#\#\#\#\#\#\#\#\#\#\#\#\#\#\#\#\#\#\#\# \[
\boldsymbol{G}_{diag} = \sigma_{G}^2\boldsymbol{I}_{(R+C)}
\] \#\#\#\#\#\#\#\#\#\#\#\#\#\#\#\#\#\#\#\#\#\#\#
\(\boldsymbol{I}_{(R+C)}\) is a \((R+C)\times(R+C)\) identity matrix.
And \(\sigma_{G}\) is a scale constant. This means that the influences
of the rows and columns are independent of each other, meaning there is
no correlation between row and column effects in the design.

Similarly, for the covariance matrix for the error term, matrix
\(\boldsymbol{R}\), I assume that it is also diagonal, indicating that
the residual errors are uncorrelated across different experimental
units. In this case we have \[
\boldsymbol{R}_{diag} = \sigma_{R}^2\boldsymbol{I}_{RC}
\] with identity matrix \(\boldsymbol{I}_{RC}\) and scale constant
\(\sigma_{R}\).

\citet{butler2013optimal} have introduced linear transformations of the
treatment parameter vector \(\boldsymbol{\tau}\) by using a
transformation matrix \(\boldsymbol{D}\), which allows for the
investigation of linear combinations of treatments. However, in this
paper, I simplify the approach by setting \(\boldsymbol{D}\) as the
identity matrix \(\boldsymbol{I}_{RC}\). This means that we focus on the
individual effects of the treatments rather than their linear
combinations.

These assumptions makes the structure of the model becomes more
straightforward, allowing us to concentrate on the direct estimation of
treatment effects while maintaining independence among the random
effects and error terms.

It's important to note that in our design, the random effect matrix
\(\boldsymbol{Z}\) remains constant during the optimization process.This
means that while the row and column effects are accounted for as random
effects, their structure does not change.

With these assumptions in place, the A-value in the context of a
row-column design dependents only on the treatment design matrix
\(\boldsymbol{X}\). This means that the primary factor influencing the
A-value is the distribution and arrangement of treatments within the
design, and it directly impacts the variance of the treatment effect
estimates. Therefore, optimizing the A-value under these assumptions
becomes a problem of optimizing the treatment distribution in the
design, ensuring that the treatments are arranged in such a way that the
variance of the estimates is minimized.

\section{Searching Strategy}\label{searching-strategy}

Before introducing the details of the searching strategy, it is
important to establish a solid foundation by proving that the minimum of
the A-value exists. The existence of the minimum A-value implies that
the A-value has a lower bound, ensuring that the process of iteration
optimizing the experimental design is not endless. As we continue to
search for smaller A-values, this guarantees that we can eventually stop
when the A-value stabilizes or a sufficient number of iterations
reached.

This allows us to conclude that we have found an optimal or near-optimal
design. Therefore, the existence of this lower bound serves as a
critical foundation for our iterative search, giving us confidence that
the optimization will converge to a solution.

\subsection{Existence of the minimum of
A-value}\label{existence-of-the-minimum-of-a-value}

To prove that the minimum of the A-value exists, we establish a
objective function, that is, the A-value function \(A(\boldsymbol{X})\)
maps design matrix for treatment \(\boldsymbol{X}\) to its A-value. It
is a function that maps the design space to \(\mathbb{R}\). \[
A: \Omega \to \mathbb{R}, \quad \boldsymbol{X} \mapsto A(\boldsymbol{X})
\] Here \(\Omega\) is the design space contains all possible design
matrix \(\boldsymbol{X}\).

Now we proof the existence of minimum value of \(\boldsymbol{X}\), where
\(\boldsymbol{X}\in\Omega\).

\begin{proof}
Recalling Equation~\ref{eq-an}

\[
\mathscr{A}=\frac{1}{n_{\tau}(n_{\tau}-1)}[n_{\tau}tr(\boldsymbol{\Lambda})-\mathbb{1}_{n_{\tau}}^\top\boldsymbol{\Lambda}\mathbb{1}_{n_{\tau}}]
\] This expression is well-defined for all valid design matrices
\(\boldsymbol{X}\). So function \(A(\boldsymbol{X})\) is well-defined.

The A-value represents the average variance of the difference treatment
effect estimates, and since variances are always positive, the A-value
is naturally bounded from below by zero. So we have \[
A(\boldsymbol{X})\geq0
\] which implies that the A-criterion is bounded below.

In experimental design, the treatment design matrix \(\boldsymbol{X}\)
can take on a finite number of possible permutations, especially in
practical row-column designs where the number of treatments and
experimental units is fixed. In a finite search space, a lower bounded
function has its minimum value.
\end{proof}

\subsection{General structure}\label{general-structure}

In \citet{piepho2018neighbor}, the optimization of NB and ED was
typically carried out under the assumption that the A-value was already
optimal or fixed. This required identifying a set of solutions that
maintained the A-value while improving the balance and distribution
properties. In addition to assuming the A-value is fixed, another
approach they used is to randomly select a design and then optimizing ED
and NB.This process would be repeated multiple times, and the design
with the best A-value will be selected.

(images)

These approach separates the optimization of ED and NB from the A-value,
while I try to merge these two process into one algorithm. I use
pairwise permutation among treatments to change treatment design during
iterations. And to avoid design with bad ED and NB, I am using some
criteria to filter the permutation, only maintain or better properties
are accepted. In this way, a row-column design that satisfies multiple
optimization requirements is achieved.

(images)

Basing on optimal design search methods share a common set of features
in exploring the design space given in \citet{butler2013optimal}, my
searching method contains following part:

\begin{enumerate}
\def\labelenumi{\arabic{enumi}.}
\item
  A calculation method for an optimal criterion for a given design
  matrix \(\boldsymbol{X}\)
\item
  An interchange policy to switch the design with in search space
  \(\Omega\)
\item
  An acceptance policy for a new design.
\item
  A stopping rule to terminate the search.
\end{enumerate}

The criterion calculation part has already been discussed earlier. We
will now introduce interchange policy of switching the design.

\subsection{Permutations and
filtering}\label{permutations-and-filtering}

We use permutations of the treatments to update the design matrix for
treatment. We randomly select two different treatments and swap them
within the design matrix during the permutation process, without making
drastic changes.

Let \(\boldsymbol{X}\) be the current design matrix for treatments,
where each row corresponds to an experimental unit, and each column
represents a treatment. Suppose we randomly select two different
treatments, \(t_i\) and \(t_j\) located in \(i\)th row and \(j\)th row
respectively. \(\boldsymbol{X}_{new}\) can be written as \[
\boldsymbol{X}_{new} = \boldsymbol{P}_{ij}\boldsymbol{X}
\] with permutation matrix defined as \[
 \boldsymbol{P}_{i j}=\left[
 \begin{array}{cccccccc}
 1 & 0 & \cdots & 0 & \cdots & 0 & \cdots & 0 \\ 
 0 & 1 & \cdots & 0 & \cdots &0 & \cdots & 0 \\ 
 \vdots & \vdots & \ddots & \vdots & & \vdots & \cdots & \vdots \\ 
 0 & 0 & \cdots & 0 & \cdots & 1 & \cdots & 0 \\
 \vdots & \vdots & & \vdots & \ddots & \vdots & & \vdots\\
 0 & 0 & \cdots & 1 & \cdots & 0 & \cdots & 0 \\ 
 \vdots & \vdots & \cdots & \vdots & & \vdots & \ddots & \vdots \\ 
 0 & 0 & \cdots & 0 & \cdots & 0 & \cdots & 1
 \end{array}\right] 
\] It is a identity matrix with \(i\)th row and \(j\)th row swapped.

When performing permutations on the design matrix, I apply checks based
on the metrics of evenness of distribution (ED) and neighbour balance
(NB). The goal is to ensure that only the permutations which improve or
at least maintain desirable values for ED and NB are accepted, while
others are filtered out.

A random permutation of treatments is generated by swapping two
different treatments in the design matrix, as described earlier using a
permutation matrix. Once the new design matrix \(\boldsymbol{X}_{new}\)
with corresponding design \(\mathcal{D}_{new}\) is created, the next
step is to evaluate the quality of the permutation by calculating the ED
and NB values for the new configuration. As afore-mentioned, we have NB
and ED criteria for \(\boldsymbol{X}_{new}\), that is \(C_{NB}'\),
\(MRS(\mathcal{D}_{new})\) and \(MCS(\mathcal{D}_{new})\). Comparing the
newly generated design matrix (offspring) with the original design
matrix (parent), We accept the new permutation, if the ED and NB values
improves or maintains them without significantly worse. In practice, we
often set a tolerance for the ED and NB values. We generally allow ED
and NB to become slightly worse, as it's not necessary for them to
strictly improve or remain unchanged in every iteration. Our goal is to
achieve a balance between ED, NB, and the A-value. For instance, ED
might increase slightly while NB decreases a little, as long as the
overall balance between the three objectives is maintained. This
approach has the added benefit of lowering the acceptance threshold for
permutations, which speeds up the algorithm during the random selection
process. For instance, we set tolerance for ED and NB as \(T_{ED}\) and
\(T_{NB}\), which are none negative numbers. Current design matrix
\(\boldsymbol{X}\) corresponding design \(\mathcal{D}\) has ED and NB
value\(C_{NB}\), \(MRS(\mathcal{D})\) and \(MCS(\mathcal{D})\). New
design matrix \(\boldsymbol{X}_{new}\) mentioned above is accepted when
\begin{equation}\phantomsection\label{eq-accp}{
\begin{align*}
&(C_{NB}'\leq C_{NB}+T_{NB}) \\
\land & (MRS(\mathcal{D}_{new})\geq MRS(\mathcal{D})-T_{ED})\\
\land & (MCS(\mathcal{D}_{new})\geq MCS(\mathcal{D})-T_{ED})
\end{align*}
}\end{equation} is true.

\subsection{Random search}\label{random-search}

The random search algorithm begins with a randomly selected design
matrix. To ensure that the search is efficient and avoids getting
trapped in local optima, we introduce the concept of step length. This
parameter determines how many permutations we consider in each
iteration. Given the computational constraints, especially when the
number of treatments or rows and columns increases, it is impractical to
check all possible permutations of a design and evaluate each for
acceptance. However, we aim to explore as many permutations as possible
to avoid falling into local optima.

At each iteration, we randomly generate a set of permutations. For each
generated permutation, we apply the filtering step check by
Equation~\ref{eq-accp}. If the permutation is not accepted, we randomly
select another one and repeat the process. This continues until we have
successfully selected a number of permutations equal to the step length.
The whole process should look like

\begin{enumerate}
\def\labelenumi{\arabic{enumi}.}
\tightlist
\item
  \textbf{Input}: Original design matrix \texttt{X}, Step length
  \texttt{s}, Tolerance for ED and NB as \texttt{t\_1} and
  \texttt{t\_2}.
\item
  Initialize \texttt{k\ =\ 1}, ED and NB value for \texttt{X} as
  \texttt{ED\_X\_row}, \texttt{ED\_X\_col} and \texttt{NB\_X}.
\item
  While \texttt{k\ \textless{}\ s}:

  \begin{itemize}
  \tightlist
  \item
    Generate a random permutation matrix \texttt{P} by selecting two
    different treatments.
  \item
    Apply permutation: \texttt{X\_new\ =\ X\ P}.
  \item
    Calculate new values of ED and NB, \texttt{ED\_X\_row\_new},
    \texttt{ED\_X\_col\_new} and \texttt{NB\_X\_new}.
  \item
    If new values satisfy
    \texttt{ED\_X\_row\_new\ \textgreater{}=\ ED\_X\_row\ -\ t\_1},
    \texttt{ED\_X\_col\_new\ \textgreater{}=\ ED\_X\_col\ -\ t\_1} and
    \texttt{NB\_X\_new\ \textless{}=\ NB\_X\ +\ t\_2}, accept and
    remember this \texttt{X\_new}.
  \item
    Increment \texttt{k}.
  \item
    If \texttt{k\ =\ s}, output all selected permutations.
  \end{itemize}
\item
  \textbf{Output}:A set of \texttt{k} permutations of original design
  matrix \texttt{X}
\end{enumerate}

The Random Search algorithm starts with a randomly selected initial
design matrix \(\boldsymbol{X}_0\), and we aim to minimize the A-value
associated with the design. We set a maximum number of iterations \(M\)
and a step length \(s\).

The process for each iteration can be described as follows:

\subsection{Simulated annealing}\label{simulated-annealing}

\bookmarksetup{startatroot}

\chapter{Results}\label{sec-results}

\bookmarksetup{startatroot}

\chapter{Discussion}\label{sec-discuss}

\bookmarksetup{startatroot}

\chapter*{References}\label{references}
\addcontentsline{toc}{chapter}{References}

\markboth{References}{References}

\renewcommand{\bibsection}{}
\bibliography{ref.bib}


\backmatter


\end{document}
